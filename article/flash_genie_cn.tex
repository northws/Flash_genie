\documentclass[11pt,a4paper]{article}
\usepackage{fontspec}
\usepackage{xeCJK}
\setmainfont{Times New Roman}
\usepackage{geometry}
\geometry{left=1in,right=1in,top=1in,bottom=1in}
\usepackage{graphicx}
\usepackage{amsmath}
\usepackage{amssymb}
\usepackage{natbib}
\usepackage{hyperref}

\title{Flash-Genie:一种面向长序列蛋白质结构生成的高效内存框架}
\author{Jiayu Wu}
\date{2026年1月}

\begin{document}
\maketitle

\begin{abstract}
本文基于Genie \citep{lin2023generatingnoveldesignablediverse}提出了Flash-Genie,一种新型蛋白质结构生成框架,旨在解决长序列蛋白质建模中的计算挑战。Flash-Genie整合了多项关键创新:用于高效注意力计算的Flash不变点注意力(Flash-IPA)、用于稳定梯度的流形约束超连接(mHC)以及将内存复杂度从$O(L^2)$降低到$O(L \cdot R)$的分解成对特征。与传统方法相比,我们的框架能实现内存节省,同时在蛋白质结构预测任务上仍然有不错的表现。
\end{abstract}

\section{引言}

蛋白质结构预测随着深度学习方法的出现经历了革命性变革,例如Genie \citep{lin2023generatingnoveldesignablediverse}通过等变扩散定向残基云来生成新颖、可设计和多样化的蛋白质结构。然而,这些方法在应用于长序列蛋白质时面临重大挑战。传统方法需要$O(L^2 \cdot C)$的内存来存储成对特征,而三角运算则需要$O(L^3)$的时间复杂度,这使得超过1024个残基的序列在现有硬件上计算难以进行。此外,Evoformer中多层的深度网络在处理长序列时经常出现梯度爆炸或消失,标准注意力机制需要显式构建$L \times L$注意力矩阵,导致二次的内存和计算成本。

Flash-Genie通过系统的优化方法应对这些挑战。我们将Flash Attention \citep{dao2022flashattentionfastmemoryefficientexact}、Flash-IPA \citep{liu2025flashinvariantpointattention}和mHC架构 \citep{xie2026mhcmanifoldconstrainedhyperconnections}与分解成对特征 \citep{ho2019axialattentionmultidimensionaltransformers}集成,以直接生成分解表示而不产生中间$L^2$张量。实现了三角运算分解,在保持类Evoformer处理表达能力的同时避免了$O(L^3)$的内存复杂度。随后引入了渐进式训练调度器和分块损失计算,允许课程学习策略以实现长序列的稳定收敛。当序列长度超过实际限制时,采用稀疏$k$NN成对选择将内存从$O(L^2)$减少到$O(L \cdot k)$。在此之后,通过轴向注意力和模型压缩技术(包括层共享和瓶颈架构)进一步优化计算效率。最后,通过分布式数据并行、序列张量并行和梯度累积机制实现大规模训练。

\section{方法论(Methodology)}

\subsection{问题形式化}

给定蛋白质序列$S = (s_1, s_2, \dots, s_L)$,其中$s_i$表示位置$i$处的氨基酸,目标是预测所有原子或主链原子的三维坐标$\mathbf{x} = (\mathbf{x}_1, \mathbf{x}_2, \dots, \mathbf{x}_L)$。传统基于扩散的方法\citep{lin2023generatingnoveldesignablediverse}将蛋白质结构空间中的生成过程建模为生成过程。核心挑战在于高效表示和处理所有残基对$(i, j)$之间的成对相互作用,这自然导致$L \times L$成对张量$\mathbf{P} \in \mathbb{R}^{L \times L \times C}$。

AlphaFold2\citep{Jumper_2021}的Evoformer模块通过交替的注意力和转换操作处理单表示$\mathbf{s} \in \mathbb{R}^{L \times C_s}$和成对表示$\mathbf{p} \in \mathbb{R}^{L \times L \times C_p}$。关键的计算瓶颈是成对表示,由于三角运算需要$O(L^2 \cdot C)$的内存和$O(L^3)$的时间。对于$L = 2048$和$C = 128$,仅成对张量(FP32)就需要约2.1 GB的内存,不包括梯度或中间激活值。

\subsection{分解成对特征}

分解成对特征模块将$O(L^2)$成对张量分解为低秩因子。给定秩$R$,成对表示重构为:

\begin{align}
    \mathbf{p}_{ij} &= \sum_{r=1}^{R} \left( \mathbf{f}_{L, i, r} \odot \mathbf{f}_{R, j, r} \right) \\
    \mathbf{f}_{L, i, r} &= \mathbf{f}_{L, i, r}^{(s)} + \mathbf{f}_{L, i, r}^{(\text{rel})} + \mathbf{f}_{L, i, r}^{(\text{tmpl})} \\
    \mathbf{f}_{R, j, r} &= \mathbf{f}_{R, j, r}^{(s)} + \mathbf{f}_{R, j, r}^{(\text{rel})} + \mathbf{f}_{R, j, r}^{(\text{tmpl})}
\end{align}

其中$\odot$表示逐元素乘法。内存复杂度从$O(L^2 \cdot C)$减少到$O(L \cdot R \cdot C)$。

分解相对位置编码计算为:

\begin{align}
    \mathbf{h}_i &= \left[ \text{Emb}_{\text{abs}}(i) ; \text{Emb}_{\text{bin}}(\text{clamp}(i - L/2)) \right] \\
    \mathbf{f}_{L, i, r}^{(\text{rel})} &= \mathbf{W}_{L}^{(\text{pos})} \mathbf{h}_i + \mathbf{b}_{\text{rel}, r} \\
    \mathbf{f}_{R, j, r}^{(\text{rel})} &= \mathbf{W}_{R}^{(\text{pos})} \mathbf{h}_j - \mathbf{b}_{\text{rel}, r}
\end{align}

分解模板特征使用注意力池化:

\begin{align}
    \mathbf{v}_{\text{diag}, i} &= \mathbf{T}_{ii} \\
    \mathbf{v}_{\text{row}, i} &= \sum_k \text{Softmax}(\mathbf{w}_q^T \mathbf{T}_{ik}) \cdot \mathbf{T}_{ik} \\
    \mathbf{v}_{\text{col}, i} &= \sum_k \text{Softmax}(\mathbf{w}_q^T \mathbf{T}_{ki}) \cdot \mathbf{T}_{ki} \\
    \mathbf{f}_{L, i, r}^{(\text{tmpl})} &= (\mathbf{W}_U \mathbf{u}_{L, i}) \cdot \sigma_r \\
    \mathbf{f}_{R, j, r}^{(\text{tmpl})} &= \mathbf{W}_V \mathbf{u}_{R, j}
\end{align}

其中$\sigma_r$是学习到的奇异值标量。

分解三角乘法更新计算为:

\begin{align}
    A_{left} &= \text{Linear}_{a}(\text{LN}(Z_{left})) \odot \sigma(\text{Linear}_{g\_a}(\text{LN}(Z_{left}))) \\
    B_{right} &= \text{Linear}_{b}(\text{LN}(Z_{right})) \odot \sigma(\text{Linear}_{g\_b}(\text{LN}(Z_{right}))) \\
    \tilde{A} &= A_{left} W_{mix\_a}^T, \quad \tilde{B} = B_{right} W_{mix\_b}^T \\
    \bar{B} &= \frac{1}{L} \sum_{k=1}^{L} \tilde{B}_{k} \\
    U_{left} &= \tilde{A} \odot \bar{B}
\end{align}

分块三角注意力聚合分解表示:

\begin{align}
    Z_{pair} &= \text{LN}\left(\sum_{r=1}^{R} Z_{left, r} + \sum_{r=1}^{R} Z_{right, r}\right) \\
    Q, K, V &= Z_{pair} W_Q, Z_{pair} W_K, Z_{pair} W_V \\
    \text{Scores}^{(m)} &= \frac{Q_{[i:i+S]} K^T}{\sqrt{d_h}} + B_{bias[i:i+S]}
\end{align}

其中$S$是块大小。

流形约束超连接(mHC)架构通过在扩展的"超空间"$\mathcal{Z} \in \mathbb{R}^{L \times n \times C}$中维护表示来解决梯度流稳定性问题:

\begin{align}
    \bar{x} &= \text{RMSNorm}(\text{Flatten}(\mathcal{Z})) \\
    \mathbf{H}_{pre} &= \sigma(\mathbf{W}_{pre} \bar{x} + b_{pre}) \in \mathbb{R}^{1 \times n} \\
    \mathbf{H}_{post} &= 2 \cdot \sigma(\mathbf{W}_{post} \bar{x} + b_{post}) \in \mathbb{R}^{n \times 1} \\
    \mathbf{A} &= \mathbf{W}_{res} \bar{x} + b_{res} \in \mathbb{R}^{n \times n} \\
    \mathbf{H}_{res} &= \text{SinkhornKnopp}(\mathbf{A})
\end{align}

Sinkhorn-Knopp迭代确保$\mathbf{H}_{res}$是双随机的(行和列和等于1),保证谱半径为1,从而通过深度网络实现稳定梯度流。更新机制如下:

\begin{align}
    s_{contracted} &= \mathbf{H}_{pre} \cdot \mathcal{S} \\
    \Delta s &= \text{IPA}(s_{contracted}, p, T, \text{mask}) \\
    \Delta \mathcal{S} &= \mathbf{H}_{post} \otimes \Delta s \\
    \mathcal{S}^{(l+1)} &= \mathbf{H}_{res} \cdot \mathcal{S}^{(l)} + \Delta \mathcal{S}
\end{align}

\subsection{渐进式训练和混合精度}

课程学习调度器在训练期间动态调整序列长度:

\begin{align}
    p &= \text{clamp}\left(\frac{t - T_{warmup}}{T_{growth}}, 0, 1\right) \\
    \alpha &= \text{schedule}(p) \quad \text{(线性、余弦或指数)} \\
    L_{curr}(t) &= \begin{cases}
        L_{min} & t < T_{warmup} \\
        \lfloor L_{min} + \alpha \cdot (L_{max} - L_{min}) \rfloor & T_{warmup} \leq t < T_{total} \\
        L_{max} & \text{否则}
    \end{cases}
\end{align}

分块损失计算避免构建完整的$L \times L$距离矩阵:

\begin{align}
    d_{ij} &= \|\mathbf{x}_i - \mathbf{x}_j\|_2, \quad \hat{d}_{ij} = \|\hat{\mathbf{x}}_i - \hat{\mathbf{x}}_j\|_2 \\
    E_{ij} &= \min\left( |\hat{d}_{ij} - d_{ij}|, \tau \right) \\
    \mathcal{L}_{dist} &= \frac{\sum_{k} \sum_{i \in \text{chunk}_k} \sum_{j=1}^{L} M_{ij} E_{ij}}{\sum_{i,j} M_{ij} + \epsilon}
\end{align}

内存复杂度从$O(L^2)$减少到$O(S \cdot L)$,其中$S$是块大小。混合精度训练通过FP16/BF16支持和动态损失缩放提供约50\%的内存节省和2-3倍训练加速。

mHC正则化损失通过多种机制稳定深度网络训练。残差平衡损失约束残差和主分支贡献之间的比率:

\begin{align}
    \rho &= \frac{\|\mathbf{F}(\mathbf{x})\|_M^2}{\|\mathbf{x}_{in}\|_M^2 + \|\mathbf{F}(\mathbf{x})\|_M^2 + \delta} \\
    \mathcal{L}_{balance} &= (\rho - \gamma)^2
\end{align}

梯度范数保持损失维持稳定的梯度流:

\begin{align}
    r &= \frac{\|\mathbf{x}_{out}\|_M}{\|\mathbf{x}_{in}\|_M + \delta} \\
    \mathcal{L}_{norm} &= \text{ReLU}(|r - 1| - \tau)^2
\end{align}

双随机惩罚强制权重矩阵的双随机属性:

\begin{align}
    \tilde{W} &= \exp(W) \\
    \mathcal{L}_{DS} &= \frac{1}{N} \sum_i (\sum_j \tilde{W}_{ij} - 1)^2 + \frac{1}{N} \sum_j (\sum_i \tilde{W}_{ij} - 1)^2
\end{align}

对于长序列($L > 1024$),自适应损失函数动态调整容差和权重:

\begin{align}
    \lambda &= \sqrt{L / L_{base}} \\
    \tau_{long} &= \max(0.1, \frac{0.2}{\lambda}) \\
    \mathcal{L}_{adaptive} &= \lambda \cdot \text{ReLU}\left( \left| \frac{\|\epsilon_{pred}\|}{\|\epsilon_{target}\|} - 1 \right| - \tau_{long} \right)^2
\end{align}

\subsection{稀疏$k$NN成对选择}

对于超长序列,稀疏$k$NN成对选择将内存从$O(L^2)$减少到$O(L \cdot k)$。混合选择策略结合了基于坐标和基于序列的邻居:

\begin{align}
    d_{ij} &= \|\mathbf{x}_i - \mathbf{x}_j\|_2 \\
    \mathcal{N}_i^{coord} &= \underset{j}{\text{argtopk}}(d_{i,:}, k_{coord}, \text{largest=False}) \\
    \mathcal{N}_i^{seq} &= \{j \mid \max(0, i - k_{seq}/2) \leq j \leq \min(L-1, i + k_{seq}/2)\} \\
    \mathcal{N}_i^{hybrid} &= \mathcal{N}_i^{coord} \cup \mathcal{N}_i^{seq}
\end{align}

内存复杂度比较:
\begin{align}
    \text{密集成对:} &\quad O(L^2 \cdot C) \\
    \text{稀疏成对($k=32$):} &\quad O(L \cdot k \cdot C) \approx \frac{1}{120} O(L^2 \cdot C) \quad (L=4096)
\end{align}

强制性局部成对确保二级结构信息的覆盖:

\begin{align}
    \mathcal{N}_i^{local} &= \{j \mid |i - j| \leq w\} \\
    \mathcal{N}_i^{final} &= \mathcal{N}_i^{knn} \cup \mathcal{N}_i^{local}
\end{align}

其中$w$是局部窗口大小。

\subsection{轴向注意力和模型压缩}

轴向注意力将二维注意力分解为顺序的一维操作:

\begin{align}
    X_{row\_view} &\in \mathbb{R}^{(B \cdot L) \times L \times C} \\
    X^{(1)} &= X + \text{Attention}_{row}(X_{row\_view}) \\
    X_{transposed} &= (X^{(1)})^T \\
    X^{(2)} &= X^{(1)} + \left( \text{Attention}_{col}(X_{transposed}) \right)^T
\end{align}

内存复杂度从$O(L^2)$减少到分块实现的$O(L)$,计算复杂度从$O(L^3)$减少到$O(L^2)$。

通过层共享的模型压缩显著减少参数。通用层共享重复应用相同的层$f_\theta$:

\begin{align}
    x^{(l)} &= f_{\theta}(x^{(l-1)}), \quad l = 1, \dots, L \\
    \text{参数} &\propto O(1 \times |\theta|)
\end{align}

交替层共享使用两组参数:

\begin{align}
    x^{(l+1)} &= \begin{cases}
        f_{\theta_{even}}(x^{(l)}) & l \equiv 0 \pmod{2} \\
        f_{\theta_{odd}}(x^{(l)}) & l \equiv 1 \pmod{2}
    \end{cases}
\end{align}

分块层共享将层分为$K$个块:

\begin{align}
    x^{(k, m+1)} &= f_{\theta_k}(x^{(k, m)}), \quad m \in [0, M-1] \\
    \text{参数} &\propto O(K \times |\theta|)
\end{align}

瓶颈层通过维度投影减少计算成本:

\begin{align}
    x_{bot} &= x W_{down} + b_{down}, \quad W_{down} \in \mathbb{R}^{d_{in} \times (d_{in}/r)} \\
    h &= \text{Operation}(x_{bot}) \\
    y &= h W_{up} + b_{up}, \quad W_{up} \in \mathbb{R}^{(d_{in}/r) \times d_{in}} \\
    \text{输出} &= \text{LayerNorm}(x + y)
\end{align}

深度可分离层进一步减少参数:

\begin{align}
    Y_{c, i} &= \sum_{k} W_{c, k}^{depth} \cdot X_{c, i+k} \\
    Z_{j, i} &= \sum_{c} W_{j, c}^{point} \cdot Y_{c, i}
\end{align}

与标准卷积相比,深度可分离层将参数从$C_{out} \times C_{in} \times K$减少到$C_{in} \times K + C_{out} \times C_{in}$。

压缩比计算为:

\begin{align}
    \text{比率} &= \frac{L \times |\theta|_{base}}{|\Theta|_{shared}}
\end{align}

对于$L=12$层,通用共享实现12倍压缩,交替共享实现6倍,分块共享(大小=4)实现4倍压缩。

\subsection{阶段5:分布式训练}

分布式数据并行(DDP)通过同步梯度更新实现多GPU训练。序列张量并行沿序列维度分割计算,允许不同GPU处理蛋白质序列的不同部分。梯度累积通过在多个微批次上累积梯度来实现大于物理GPU内存的有效批量大小:

\begin{align}
    \text{梯度}_{total} &= \sum_{m=1}^{M} \text{梯度}_{micro\_batch}^{(m)}
\end{align}

这种方法实现稳定的大批量训练,同时保持类似于单批次优化的梯度统计。

有效批量大小随GPU数量线性缩放:

\begin{align}
    \text{eff}_{bs} &= \text{eff}_{bs\_per\_GPU} \times n_{GPUs}
\end{align}

多GPU训练在保持统计等价于单GPU训练的同时减少壁钟时间,这通过正确同步模型参数和梯度来实现。

\subsection{mHC集成}

mHC成对变换层通过收缩-变换-扩展循环处理成对表示:

\begin{align}
    \tilde{Z}_{ij} &= H_{pre, ij} \cdot \mathcal{Z}_{ij} = \sum_{k=1}^{n} H_{pre, ij}^{(k)} \cdot z_{ij}^{(k)} \\
    \Delta Z_{ij} &= F(\tilde{Z}_{ij}) \quad \text{(标准Evoformer操作)} \\
    \Delta \mathcal{Z}_{ij} &= H_{post, ij} \otimes \Delta Z_{ij} \\
    \mathcal{Z}_{ij}^{(l+1)} &= H_{res, ij} \cdot \mathcal{Z}_{ij}^{(l)} + \Delta \mathcal{Z}_{ij}
\end{align}

mHC结构网络将超连接应用于单表示:

\begin{align}
    s_{contracted, i} &= \mathbf{H}_{pre, i} \cdot \mathcal{S}_i \\
    \Delta s_{ipa} &= \text{LayerNorm}(\text{Dropout}(\text{IPA}(s_{contracted}, p, T, \text{mask}))) \\
    \Delta s_{total} &= \text{Transition}(\Delta s_{ipa}) \\
    \Delta \mathcal{S}_i &= \mathbf{H}_{post, i} \otimes \Delta s_{total} \\
    \mathcal{S}_i^{(l+1)} &= \mathbf{H}_{res, i} \cdot \mathcal{S}_i^{(l)} + \Delta \mathcal{S}_i
\end{align}

主干更新使用收缩的特征:

\begin{align}
    s_{for\_bb} &= \text{Contract}(\mathcal{S}^{(l+1)}) \\
    T^{(l+1)} &= T^{(l)} \circ \text{BackboneUpdate}(s_{for\_bb})
\end{align}

mHC Flash结构网络结合mHC与Flash-IPA以实现最佳内存效率:

\begin{align}
    s_{in, i} &= \mathbf{H}_{pre, i} \cdot \mathcal{S}_i \\
    b_{ij} &\approx \text{Linear}(Z_{fac1, i} \odot Z_{fac2, j}) \\
    s_{ipa} &= \text{FlashAttention}(Q, K, V, \text{BiasFactors}) \\
    \Delta s &= \text{Transition}(\text{LayerNorm}(\text{Dropout}(s_{ipa}))) \\
    \mathcal{S}_i^{(l+1)} &= \mathbf{H}_{res, i} \cdot \mathcal{S}_i^{(l)} + \mathbf{H}_{post, i} \otimes \Delta s_i \\
    T^{(l+1)} &= T^{(l)} \circ \text{BackboneUpdate}(s_{ipa})
\end{align}

内存复杂度比较:
\begin{align}
    \text{标准IPA:} &\quad O(L^2 \cdot H) \\
    \text{Flash-IPA:} &\quad O(L \cdot H)
\end{align}

mHC和Flash-IPA的结合通过双随机混合实现梯度稳定性,通过Flash Attention算法实现线性内存缩放。

\section{局限性与未来工作}

尽管Flash-Genie在内存效率方面取得了显著改进,但一些局限性仍然存在。首先,稀疏$k$NN成对选择策略虽然将内存复杂度降低到$O(L \cdot k \cdot C)$,但当邻居数量$k$较小时,可能无法完全捕获残基的蛋白质中的非常长程依赖关系。其次,mHC中的动态投影矩阵和分解表示引入了额外的计算开销,对于二次缩放不构成问题的短序列,这可能抵消一些内存节省。另外,尽管mHC提供了出色的梯度稳定性,但成对特征的内存复杂度仍然保持在$O(L^2 \cdot n \cdot C)$,限制了$n$的实际使用大选,单表示可使用$n=4$,而成对表示的实际应用可能只能使用$n=2$。从数据集的方向来看,训练数据需求仍然很大,渐进式训练课程的有效性取决于多样化长序列训练样本的可用性。更重要的FlashAttention实现通常将头维度严格限制为256,当超过 256 时,GPU 甚至无法在一个线程块(Thread Block)的共享内存(SRAM)中放下最小的有效计算单元,导致算法无法运行或效率极低,这限制了分解秩,并可能限制密集成对表示的近似质量。最后,用于双随机矩阵的Sinkhorn-Knopp迭代在每层引入了额外的计算成本,尽管具有有益的梯度属性,但对于非常深的网络来说可能变得显著。

未来,我们计划探索更先进的稀疏注意力模式,以更好地捕获长程依赖关系,同时保持线性内存复杂度。我们还将研究更高效的Sinkhorn-Knopp近似,以减少迭代开销。此外,我们将探索动态$mHC$参数$n$根据序列长度自动调整的可能性,以在不同序列长度上实现最佳效率-性能权衡。最后,我们将把框架扩展到支持多链蛋白质复合物的建模。

\bibliographystyle{plainnat}
\bibliography{references}

\end{document}
